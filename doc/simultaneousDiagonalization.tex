\documentclass[aspectratio=169,10pt]{beamer}

%\documentclass[11pt]{article}
%\usepackage{beamerarticle}
%\usepackage[hidelinks]{hyperref}

\usetheme[height=1cm]{Rochester}

    \usepackage[T1]{fontenc}
    \usepackage{fourier}
    \usepackage{graphicx}
    \usepackage{caption}
    \usepackage{adjustbox} % Used to constrain images to a maximum size 
    \usepackage{xcolor} % Allow colors to be defined
    \usepackage{geometry} % Used to adjust the document margins
    \usepackage{amsmath} % Equations
    \usepackage{amssymb} % Equations

\usepackage{natbib}
    \usepackage{indentfirst}
    \usepackage{minted}
    \usemintedstyle{tango}
    \usefonttheme{serif}
    
\title[Simultaneous diagonalization]{Simultaneous diagonalization and its C++ implementation}
\author{Yu Zhai \\
	\texttt{yuzhai@mail.huiligroup.org}}
	\institute[ITC@JLU]{Institute of Theoretical Chemistry, Jilin University, Changchun, China. }
	
	\usepackage{color,soul}
	%\renewcommand{\emph}[1]{\hl{#1}}
	
    \begin{document}  
    	\begin{frame}
    \maketitle
    \end{frame}


\begin{frame}{Introduction}

    \begin{block}{}
\ldots If they (observables) do commute there exist so
many simultaneous eigenstates that they form a complete set \ldots

\flushright
P. A. M. Dirac, \textit{The Principles of Quantum Mechanics}
    \end{block}

Thus we do need an
algorithm to obtain the simultaneous eigenstates of two (or more)
matrices, 
which is described in References~\cite{Dawes2004}, \cite{BunseGerstner1993} and \cite{Cardoso1996}.

    Consider a set of matrices $\{\mathbf{A}\}$ of $N\times N$ complex matrices.
In this work, however, I prefer limit us to real symmetric matrices, 
with which are easier to deal.

When the matrices in $\{\mathbf{A}\}$ are \emph{normal commuting
	matrices}, their off-diagnoal terms can be set to zero by \emph{one} unitary
transform, thus simultaneously diagonalizing the set $\{\mathbf{A}\}$.

\end{frame}


\begin{frame}[allowframebreaks]{Mathematical description of the problem}

    Define 
\begin{equation}
\mathrm{off}(\mathbf{A})\equiv\sum_{0 \leq i \ne j < N}A_{ij}^2,
\end{equation}
where $A_{ij}$ denotes the row $i$ column $j$ entry of matrix $\mathbf{A}$.

\inputminted[linenos,firstline=21, lastline=29, breaklines]{cpp}{../simultaneousDiagonalization.cc}
   
   \framebreak
   
   
    Simultaneous diagonalization may be obtained by minimizing the composite
objective
\begin{equation}
\mathrm{offsum}(\{\mathbf{A}\})\equiv\sum_{\mathbf{A}\in \{\mathbf{A}\}}\mathrm{off}(\mathbf{A}).
\end{equation}

\inputminted[linenos,firstline=31, lastline=37, breaklines]{cpp}{../simultaneousDiagonalization.cc}

\end{frame}
         
         \begin{frame}[allowframebreaks]{Extended Jacobi technique} 
    The extended Jacobi technique for simultaneous diagonalization
constructs \(\mathbf{U}\) as a product of plane rotations globally applied to all
\(\mathbf{A}\in\{\mathbf{A}\}\).

    It is desired, for each choice of \(i\neq j\), to find the \(c\) and
\(s\) so that \[
O(c,s)\equiv\sum_{\mathbf{A}\in\{\mathbf{A}\}}\mathrm{off}(\mathbf{R}'(i,j,c,s)\mathbf{A}\mathbf{R}(i,j,c,s))
\] is minimized.
$'$ in this document means adjoint (conjugate transpose). 
\only<article>{\footnote{The \texttt{'} notation is used in Julia, by the way.}}

    Let us define yet another \(2\times 2\) real symmetric matrix \(\mathbf{G}_{ij}\) for
\((i,j)\) \begin{equation}
\mathbf{G}_{ij}\equiv \sum_{\mathbf{A}\in\{\mathbf{A}\}}\mathbf{h}(\mathbf{A})\mathbf{h}'(\mathbf{A}),
\end{equation}
    where \begin{equation}
\mathbf{h}(\mathbf{A})\equiv[A_{ii}-A_{jj},\, A_{ij}+A_{ji}]'.
\end{equation}

\framebreak

\inputminted[linenos,firstline=39, lastline=50, breaklines]{cpp}{../simultaneousDiagonalization.cc}

\framebreak
            
    Denote \(\mathbf{R}(i,j,c,s)\) the rotation matrix equal to the identity
matrix but for the following entries: 
\begin{equation}
    \begin{pmatrix}
R_{ii} & R_{ij} \\ R_{ji} & R_{jj}\end{pmatrix}=\begin{pmatrix}
c & -s \\ s & c
\end{pmatrix} \text{ with } c,s\in\mathbb{R}\text{ and } c^2+s^2=1.
\end{equation}

    Thus we have a theorem, whose proof is not given here: Under
constraint \(c^2+s^2=1\), \(O(c,s)\) is minimized at \[
c=\sqrt{\frac{x+r}{2r}}\quad s=\frac{y}{\sqrt{2r(x+r)}}\quad r=\sqrt{x^2+y^2},
\] where \((x, y)'\) is any eigenvector associated to the largest
eigenvalue of \(G\).  If the $(x,y)'$ are normalized, then $r =1$, formula above turns to
\begin{equation}
c=\sqrt{\frac{x+1}{2}}\quad s=\frac{y}{\sqrt{2(x+1)}}.
\end{equation}
            
            \framebreak
            \inputminted[linenos,firstline=52, lastline=65, breaklines]{cpp}{../simultaneousDiagonalization.cc}
   
   \framebreak         
    Then we have the Jacobi-like iteration scheme.

\inputminted[linenos,firstline=67, lastline=78, breaklines]{cpp}{../simultaneousDiagonalization.cc}
\ldots \, \ldots
\inputminted[linenos,firstline=87, lastline=87, breaklines]{cpp}{../simultaneousDiagonalization.cc}


\end{frame}

\begin{frame}[allowframebreaks=1]{C++ syntax related}
To make the program work, the class is defined as

\inputminted[linenos,firstline=6,lastline=17,breaklines]{cpp}{../simultaneousDiagonalization.h} 
\ldots\,\ldots
\inputminted[linenos,firstline=27,lastline=27,breaklines]{cpp}{../simultaneousDiagonalization.h} 
\framebreak
And its constructor

\inputminted[linenos,firstline=4, lastline=19, breaklines]{cpp}{../simultaneousDiagonalization.cc}
\end{frame}

\begin{frame}{Unit test}
\begin{columns}
	\begin{column}{0.3\textwidth}
Three tests are designed while here I only show the last one.
\begin{itemize}
	\item Both matrices $\mathbf{A}$ and $\mathbf{B}$ are generated from eigenvalues;
	\item They are both degenerated;
	\item Use the method presented \textsc{can} give reasonable solution. 
\end{itemize}
	\end{column}
	\begin{column}{0.7\textwidth}
		Results:
\scriptsize
\inputminted[linenos, breaklines]{text}{../test3.txt}
	\end{column}
\end{columns}

\end{frame}

\begin{frame}
\centering\itshape
{\large Thank you for your attention. }

\vspace{0.5in}

\small\today
\end{frame}
 
 \begin{frame}{Reference}
 \bibliographystyle{unsrt}
 \bibliography{/home/yuzhai/Library/000ref}
\end{frame}


    \end{document}
